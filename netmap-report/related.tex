\section{Related Work}
\label{s:related}


Network Diagnostic Tool (NDT)~\cite{NDT} collects performance data over wireless links using a client/server architecture. The server consists of a webserver and an analysis engine. The client communicates with this enhanced server to perform diagnostic tests including web page request and the server collects the resulting measurements and attempts to identify the cause of performance issues. The primary goal of ndt is to identify network performance issues, which occur close to users (eg. incorrectly set TCP buffers). The server locations are all known (clients connect to one of the closest servers) and servers collect data making it easier to measure certain statistics such as one-way latency.


Dasu~\cite{dasu-sanchez} is a measurement platform for the Internet's edge. Dasu can support broadband characterization as well as internet measurement experiements. They design Dasu for the edge of the Internet so measurements reflect end users' views of the services they are using. Dasu also does not use dedicated infrastructures for experimentation. Instead, they use an incentive model to make sure it is widely adopted at the edge of the Internet. Dasu has a distributed set of clients and a set of management services. Clients perform measurements and the management services configure clients,  perform administration of experiments, and handle data collection. Dasu provides a programming interface that is flexible to run many kinds of tests (when-then model where condition dictates type of test). 

MIST (mobile internet services test), a distributed platform for measuring cellular network performance of users with hopes of aiding mobile application developers. MIST is a mobile app connected to server back-end. Communication between the mobile application on the user's device and the servers are performed to measure characeteristics of the cellular networks, including latency, jitter, throughput, etc. The database at the server saves the measurement data along with mobile device info/configuration from the test. Perk is that MIST can be deployed on top of mobile devices (don't have to change cell network infrastructure). App first collects info about mobile device, service provider, and test location. Mobile app connects to closest server to get most accurate measurements. Then app sends packets of set byte-size to analyze uplink and downlink latency, throughput, and timeouts. Difference from ours is that MIST is an app designed to get such measurements for mobile app developers (we wrap measurement collection in a game so it is not specifically used for this purpose). 

Balachandran et al.~\cite{Balachandran-sigmetrics} capture a workload at a large conference and analyze it to understand user behavior and network performance. They collect a continuous trace of SNMP data from all APs in the conference main room as well as a tcpdump trace of network-level headers of packets going through switch which all APs connected to. This provided aggregate packet level statistics of all traffic passing through these APs at the link, network and transport layers. Also, they obtained information about the users associated with the APs such as their MAC addresses, SNR, and effective throughput. They inferred the number of distinct wireless users by counting the number of distinct MAC addresses in packets passing through the APs present. The primary goal was to analyze user behavior in terms of mobility, application popularity, data rates, etc. In terms of network performance, they measure the aggregate offered load for each of the APs and observe the bursty behavior. They also measure packet errors by using the SNMP trace where APs count the total number of packets transmitted and received, and the number of packets in error (account for inbound packets that could not be delivered to higher layer and outbound packets that cant be transmitted due to channel). 

VISUM~\cite{VISUM} is a framework for wireless network monitoring that uses set of agents within network (scales better than centralized) to monitor network devices and store info at repositories. VISUM also visualizes the data into real-time statistical graphs and interactive network topology maps. They target single-hop wireless networks.  Thus, VISUM relies on a distributed architecture (agents at different locations) to monitor large scale wireless networks. Agents collect measurement info from network devices using SNMP and store the data in a centralized repository (data stored per device using device OID).  
